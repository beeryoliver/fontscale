\ProvidesFile{fontscale-table-classic.tex}[2025-04-04]

\begin{table}
  \centering
  \caption
    {%
      The font size of each font size command from \cs{tiny} to \cs{Huge} in units of \unit{pt}, \unit{bp}, \unit{dd}, or \unit{nd} when using a classic typographic scale.%
      \label{table:classic}%
    }
  \bigskip
  \begin{tblr}
    {
        colspec=
          {
            l
            Q[si={table-format=3.2},c]
            Q[si={table-format=3.2},c]
            Q[si={table-format=3.2},c]
          }
      , cell{1}{2-Z}={guard}
      , cell{2-Z}{1}={cmd=\cs}
    }
    \toprule
      font size command
      & {\key{classic-10pt} \\ \key{classic-10bp} \\ \key{classic-10dd} \\ \key{classic-10nd}}
      & {\key{classic-11pt} \\ \key{classic-11bp} \\ \key{classic-11dd} \\ \key{classic-11nd}}
      & {\key{classic-12pt} \\ \key{classic-12bp} \\ \key{classic-12dd} \\ \key{classic-12nd}} \\
    \midrule
      tiny         &  6 &  7 &  8 \\
      scriptsize   &  7 &  8 &  9 \\
      footnotesize &  8 &  9 & 10 \\
      small        &  9 & 10 & 11 \\
      normalsize   & 10 & 11 & 12 \\
      large        & 11 & 12 & 14 \\
      Large        & 12 & 14 & 16 \\
      LARGE        & 14 & 16 & 18 \\
      huge         & 16 & 18 & 21 \\
      Huge         & 18 & 21 & 24 \\
    \bottomrule
  \end{tblr}
\end{table}